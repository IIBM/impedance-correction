%|//////////////////////////////////////|\\\\\\\\\\\\\\\\\\\\\\\\\\\\\\\\\\\\\|%
%|//////////////////////////| Objetivo del Proyecto |\\\\\\\\\\\\\\\\\\\\\\\\\|%
%|//////////////////////////////////////|\\\\\\\\\\\\\\\\\\\\\\\\\\\\\\\\\\\\\|%
\section{Objetivo del Proyecto}
    El objetivo es la respuesta a la pregunta “¿para qué?”; tiene que satisfacer
    alguna necesidad del usuario.



%|//////////////////////////////////////|\\\\\\\\\\\\\\\\\\\\\\\\\\\\\\\\\\\\\|%
%|////////////////////////| Descripción del Proyecto |\\\\\\\\\\\\\\\\\\\\\\\\|%
%|//////////////////////////////////////|\\\\\\\\\\\\\\\\\\\\\\\\\\\\\\\\\\\\\|%
\section{Descripción del Proyecto}
    En esta sección se responde la pregunta “¿cómo?”. Aquí se enumeran las
    prestaciones, las funciones y el comportamiento o uso del equipo.



%|//////////////////////////////////////|\\\\\\\\\\\\\\\\\\\\\\\\\\\\\\\\\\\\\|%
%|///////////////| Características y Especificaciones Mínimas |\\\\\\\\\\\\\\\|%
%|//////////////////////////////////////|\\\\\\\\\\\\\\\\\\\\\\\\\\\\\\\\\\\\\|%
\section{Características y Especificaciones Mínimas}
    Las especificaciones acotan las bondades del equipo. Deben listarse los
    rangos de funcionamiento o requerimientos externos (por ejemplo: tensión de
    alimentación, consumo, temperatura de funcionamiento, protocolos de
    comunicaciones, rangos de medición, etc.)



%|//////////////////////////////////////|\\\\\\\\\\\\\\\\\\\\\\\\\\\\\\\\\\\\\|%
%|/////////////////////////| Periféricos Principales |\\\\\\\\\\\\\\\\\\\\\\\\|%
%|//////////////////////////////////////|\\\\\\\\\\\\\\\\\\\\\\\\\\\\\\\\\\\\\|%
\section{Periféricos Principales}
    Los periféricos son los elementos ajenos al microcontrolador con los que
    interactúa el equipo y deben estar claramente definidos (por ejemplo:
    motores, display, teclado, puertos de comunicación, sensores, etc.)



%|//////////////////////////////////////|\\\\\\\\\\\\\\\\\\\\\\\\\\\\\\\\\\\\\|%
%|////////////////| Diagrama en Bloques Preliminar (hardware) |\\\\\\\\\\\\\\\|%
%|//////////////////////////////////////|\\\\\\\\\\\\\\\\\\\\\\\\\\\\\\\\\\\\\|%
\section{Diagrama en Bloques Preliminar (hardware)}
    El esquema general de interconexión de todos los dispositivos importantes se
    representa mediante un diagrama en bloques.



%|//////////////////////////////////////|\\\\\\\\\\\\\\\\\\\\\\\\\\\\\\\\\\\\\|%
%|/////////////////| Diagrama de Flujo Preliminar (firmware) |\\\\\\\\\\\\\\\\|%
%|//////////////////////////////////////|\\\\\\\\\\\\\\\\\\\\\\\\\\\\\\\\\\\\\|%
\section{Diagrama de Flujo Preliminar (firmware)}
    El diagrama de flujo ilustra de manera general la interacción entre los
    distintos bloques (o rutinas) de código.



%|//////////////////////////////////////|\\\\\\\\\\\\\\\\\\\\\\\\\\\\\\\\\\\\\|%
%|/////////////////////////| Plan de Trabajo (Gantt) |\\\\\\\\\\\\\\\\\\\\\\\\|%
%|//////////////////////////////////////|\\\\\\\\\\\\\\\\\\\\\\\\\\\\\\\\\\\\\|%
\section{Plan de Trabajo (Gantt)}
    Un diagrama de Gantt es la propuesta de distribución de tiempos y recursos a
    lo largo del proyecto. Es necesario elaborar un plan inicial y ajustarlo en
    forma continua para no perder de vista el objetivo final, los hitos y el
    camino crítico para alcanzarlo.



%|//////////////////////////////////////|\\\\\\\\\\\\\\\\\\\\\\\\\\\\\\\\\\\\\|%
%|////////////////| Listado de Componentes y Costos Estimados |\\\\\\\\\\\\\\\|%
%|//////////////////////////////////////|\\\\\\\\\\\\\\\\\\\\\\\\\\\\\\\\\\\\\|%
\section{Listado de Componentes y Costos Estimados}
    En esta instancia se pretende un listado de los componentes más
    significativos con sus costos aproximados más una previsión general de
    elementos menores. Lo que se busca es considerar la viabilidad económica
    del proyecto.



%|//////////////////////////////////////|\\\\\\\\\\\\\\\\\\\\\\\\\\\\\\\\\\\\\|%
%|///////////////////////| Factores Críticos de Éxito |\\\\\\\\\\\\\\\\\\\\\\\|%
%|//////////////////////////////////////|\\\\\\\\\\\\\\\\\\\\\\\\\\\\\\\\\\\\\|%
\section{Factores Críticos de Éxito}
    Es importante analizar cuáles son los riesgos potenciales más importantes
    que podrían dificultar la realización del proyecto. Una vez identificados,
    conviene hacerles un seguimiento cercano para evitar contratiempos.
